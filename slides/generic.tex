\documentclass{beamer}
\usepackage{fontspec}
\usefonttheme{serif}
%% \setmainfont{Liberation Serif}
%% \setmainfont{Ubuntu Mono}
\setmonofont{Ubuntu Mono}

\usepackage{listings}

\definecolor{dkgreen}{rgb}{0,0.6,0}
\definecolor{gray}{rgb}{0.5,0.5,0.5}
\definecolor{mauve}{rgb}{0.58,0,0.82}

%% \lstset{language=Haskell}

\lstset{language=Haskell,
  showstringspaces=false,
  columns=flexible,
  basicstyle={\small\ttfamily},
  keywordstyle=\color{black},
  commentstyle=\color{dkgreen},
  %% stringstyle=\color{mauve},
  breaklines=true,
  breakatwhitespace=true,
  tabsize=3
}

\mode<presentation>
{
  %% \usetheme{Madrid}
  % or ...
  %% \setbeamercovered{transparent}
  % or whatever (possibly just delete it)
}

\usepackage[english]{babel}
%% \usepackage[latin1]{inputenc}
%% \usepackage{times}
%% \usepackage[T1]{fontenc}
% Or whatever. Note that the encoding and the font should match. If T1
% does not look nice, try deleting the line with the fontenc.


\title{Introduction to Elm}

\subtitle{Radically simpler GUI programming}

\author[Ayman Osman]{
  Ayman Osman
  \\ \texttt{aymano.osman@gmail.com}
  \\ @mrozuman
}


% Delete this, if you do not want the table of contents to pop up at
% the beginning of each subsection:
%% \AtBeginSubsection[]
%% {
%%   \begin{frame}<beamer>{Outline}
%%     \tableofcontents[currentsection,currentsubsection]
%%   \end{frame}
%% }


% If you wish to uncover everything in a step-wise fashion, uncomment
% the following command: 

%\beamerdefaultoverlayspecification{<+->}


\begin{document}

\begin{frame}
  \titlepage
\end{frame}

\begin{frame}{Outline}
  \tableofcontents
\end{frame}


\section{A Brief History}

\subsection{aaa}

\begin{frame}{A Brief History of FRP}
  \begin{enumerate}
  \item Connal Elliott
  \item Evan Czaplicki
  \end{enumerate}
\end{frame}


\begin{frame}
  \frametitle{bbb}

  \begin{columns}[t]
    \column{.5\textwidth}
    \begin{block}{A block}
      qqq
    \end{block}

    \begin{block}{B Block}
      ppp
    \end{block}
  \end{columns}
\end{frame}


%% \begin{frame}[fragile]

%%   \begin{semiverbatim}
%%     \uncover<1->{foo :: m a -> (a -> b) -> m b}
%%     \uncover<2->{foo m f = }
%%     \uncover<3->{  do x <- m }
%%     \uncover<2->{  return (f x)}
%%   \end{semiverbatim}
  
%% \end{frame}


\section{Syntax Crash Coursess}
\subsection{Named Functions}

\frame %% pause

\begin{frame}[fragile]
  \frametitle{\insertsection: Named Functions}
    \begin{lstlisting}
      -- JavaScript
      function square(n) {
        return n * n;
      }

      -- Elm
      square n = n^2
    \end{lstlisting}
\end{frame}

\subsection{Anonymous Functions}

\begin{frame}[fragile]
  \frametitle{\insertsection: Anonymous Functions}
    \begin{lstlisting}
      -- JavaScript
      var f = function(x) { return x + 1; };
      var g = (x) => x + 1;

      -- Elm
      f = \x -> x + 1
    \end{lstlisting}
\end{frame}

\subsection{Currying}

\begin{frame}[fragile]
  \frametitle{\insertsection: Currying}

    \begin{lstlisting}
      add x y = x + y
      -- Is equivalent to
      add = \ x -> (\y -> x + y)

      -- So the function `addThree` can be defined as
      addThree = add 3

      addThree 4 == 7 -- True
    \end{lstlisting}


    \pause

    You can think of it as:

    \begin{quote}
      if you give a function too few arguments, it will give you back a function
      that expects the rest of the arguments before giving you the answer.
    \end{quote}
\end{frame}

\section*{Summary}

\begin{frame}{Summary}

  % Keep the summary *very short*.
  \begin{itemize}
  \item
    The \alert{first main message} of your talk in one or two lines.
  \item
    The \alert{second main message} of your talk in one or two lines.
  \item
    Perhaps a \alert{third message}, but not more than that.
  \end{itemize}
  
  % The following outlook is optional.
  \vskip0pt plus.5fill
  \begin{itemize}
  \item
    Outlook
    \begin{itemize}
    \item
      Something you haven't solved.
    \item
      Something else you haven't solved.
    \end{itemize}
  \end{itemize}
\end{frame}



% All of the following is optional and typically not needed. 
\appendix
\section<presentation>*{\appendixname}
\subsection<presentation>*{For Further Reading}

\begin{frame}[allowframebreaks]
  \frametitle<presentation>{For Further Reading}
    
  \begin{thebibliography}{10}
    
  \beamertemplatebookbibitems
  % Start with overview books.

  \bibitem{Author1990}
    A.~Author.
    \newblock {\em Handbook of Everything}.
    \newblock Some Press, 1990.
 
    
  \beamertemplatearticlebibitems
  % Followed by interesting articles. Keep the list short. 

  \bibitem{Someone2000}
    S.~Someone.
    \newblock On this and that.
    \newblock {\em Journal of This and That}, 2(1):50--100,
    2000.
  \end{thebibliography}
\end{frame}

\end{document}

