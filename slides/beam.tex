\documentclass{beamer}
\usepackage{listings}

\definecolor{dkgreen}{rgb}{0,0.6,0}
\definecolor{gray}{rgb}{0.5,0.5,0.5}
\definecolor{mauve}{rgb}{0.58,0,0.82}

\lstset{frame=tb,
  language=Java,
  aboveskip=3mm,
  belowskip=3mm,
  showstringspaces=false,
  columns=flexible,
  basicstyle={\small\ttfamily},
  numbers=none,
  numberstyle=\tiny\color{gray},
  keywordstyle=\color{blue},
  commentstyle=\color{dkgreen},
  stringstyle=\color{mauve},
  breaklines=true,
  breakatwhitespace=true,
  tabsize=3
}

\begin{document}

\title{Foo Title}   
\author{Ayman Osman} 
\date{\today} 

\frame{\titlepage} 

\begin{frame}[fragile]
  \frametitle{lol}

  \begin{lstlisting}
    a + b;

    class Foo() {

      foo(Bar bar) {
        bar.show();
      }

    }
  \end{lstlisting}
  
\end{frame}

\frame[label=example]{
  \frametitle{Example Frame}
  \includegraphics[width=.5\textwidth]{evan.png}
}

\frame{\frametitle{Table of contents}\tableofcontents}

\againframe{example}

\begin{frame}
  \frametitle{There Is No Largest Prime Number}
  \framesubtitle{The proof uses \textit{reductio ad absurdum}.}
  \begin{theorem}
    There is no largest prime number.
  \end{theorem}
  \begin{proof}
    \begin{enumerate}
    \item<1-| alert@1> Suppose $p$ were the largest prime number.
    \item<2-> Let $q$ be the product of the first $p$ numbers.
    \item<3-> Then $q+1$ is not divisible by any of them.
    \item<1-> But $q + 1$ is greater than $1$, thus divisible by some prime
      number not in the first $p$ numbers.\qedhere
    \end{enumerate}
  \end{proof}
\end{frame}

\end{document}

